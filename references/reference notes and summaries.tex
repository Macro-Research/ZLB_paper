\documentclass[12pt,reqno]{article}
\usepackage[utf8x]{inputenc}
\usepackage{graphicx}
\usepackage[table]{xcolor}
\graphicspath{ {figures/} }
\usepackage[margin=0.9in]{geometry}
\usepackage{amssymb}
\usepackage[english]{babel}
\usepackage[T1]{fontenc}
\usepackage{amsmath,amsfonts,amssymb}
\usepackage{pdflscape}
\usepackage{natbib}
\usepackage{verbatim}
\usepackage{gensymb}

\usepackage{bm}
\usepackage{changepage}
%\usepackage{amsart}
\usepackage{lscape}
\usepackage{soul}
\usepackage{float}
\usepackage{rotating}
\usepackage{graphicx}
\usepackage{setspace}
\usepackage[color]{changebar}
\linespread{.5}
\numberwithin{equation}{section}
\usepackage{appendix}
\usepackage{authblk}
\usepackage{hyperref}
\usepackage[most]{tcolorbox}



\hypersetup{colorlinks,allcolors=blue}
\title{Adaptive Learning at the Zero Lower Bound}
\author{Tolga Ozden}
\begin{document}

\noindent
\textbf{Expectations, Deflation Traps and Macroeconomic Policy, 2009, Evans \& Honkapohja} \\

-Assumptions in the non-linear framework already are: point forecasts (distributions do not matter), representative agents and ricardian households (i.e. households understand 
government spending rule and incorporate that in their expectations.)\\
-monetary policy is transparent and clear--> that's why agents take it into account in their expectation formation rule. \\
-target steady-state is locally determinate, low inflation steady-state is locally indeterminate.

\end{document}